
\documentclass[12pt,onesided,fullpage]{amsart}
\usepackage{amsfonts}
\usepackage{amsmath}
\usepackage{amssymb}
\usepackage{array,amsmath,graphicx,psfrag,amssymb,subfigure,tabularx,booktabs}
\usepackage{mparhack}
\usepackage{setspace}
\usepackage{natbib}
\usepackage{multicol}
\usepackage{dcolumn}
\usepackage{hyperref}
\usepackage{multirow}
\usepackage{color}
\usepackage[top=3cm, bottom=3cm, left=2.3cm, right=2.3cm]{geometry} 
%\pdfpagewidth=8.5in % for pdflatex
%\pdfpageheight=11in % for pdflatex
\pagestyle{plain}
\setcitestyle{authordate,round,semicolon,aysep={,},yysep={,}}
\bibpunct[:]{(}{)}{;}{a}{,}{,}

\doublespacing
%\graphicspath{/Users/cassydorff/Dropbox/ICEWS_CRISP_RA/AJPS/manuscript_revised/graphics/}

\graphicspath{{graphics/}}
\begin{document}




\bibliographystyle{ChicagoReedWeb} %chicago}
%\doublespacing
\title[Games on Networks: Thailand]{Modeling, Computing and Evaluating Endogeneous Networks: Understanding International Cooperation}

\author{Max Gallop}
\address{Max Gallop: Department of Political Science}
\curraddr{Duke University, Durham, NC, 27708, USA}
\email{max.gallop@duke.edu}
 
\author{Michael D. Ward}
\address{Michael D. Ward: Department of Political Science}
\curraddr{Duke University, Durham, NC, 27708, USA}
\email{michael.d.ward@duke.edu}
\thanks{This project was undertaken in the framework of an initiative funded by the Information Processing Technology Office of the Defense Advanced Research Projects Agency aimed at producing models to provide an Integrated Crisis Early Warning Systems (ICEWS) for decision makers in the U.S. defense community. }.
\\
%\href{http://dvn.iq.harvard.edu/dvn/dv/ajps} %%%%% 


\date{\today,~~Final Version}
\maketitle

\begin{abstract}
The rise of social network analyses in the social sciences has allowed empirical work to better account for interdependencies among actors and actions. However, this work has been, to a large extent, descriptive and treated the links as exogeneous and immutable. However, in many cases the actions in these networks are things like alliances or trade--phenomena that are the outcome variables for programs of social scientific research. We propose to account for both interdependencies and the endogeneous nature of networks by incorporating formal theory into the formation of social networks. We discuss algorithms for finding the equilibrium of these networks computationally as well as ways to compare the theoretical networks to observed ones, in order to evaluate the fit of the theory. We then apply these methods to the study of international cooperation--a subject where both the interdependencies and purposive nature of actors must be accounted for.
\end{abstract}
 

\section{Modeling Endogeneous Networks}
\subsection{Equilibrium Concept}
\subsection{Computational Algorithm}

\section{An Example: Modeling International Cooperation}
\subsection{Literature on International Cooperation}
\subsection{Our Model of Cooperation}
\subsection{Data}
\subsection{Description of Equilibrium Network[s]}

\section{Testing the Endogeneous Networks}
\subsection{Measures of Network Comparison}
\subsection{Observed Cooperation}
\subsection{Latent Cooperation--Do we need this one?}
\subsection{Data Analysis}

\section{Conclusions}

\end{document}\bye



%\end{document}
