\documentclass{beamer}
\usetheme{Madrid}
%\usetheme{Berlin}
%\usepackage{german}
%\usepackage[latin1]{inputenc}
%\usepackage[T1]{fontenc}
\usepackage{hyperref}
\usepackage{booktabs}
\usepackage{graphicx}
\usepackage{epsfig}
\usepackage{listings}
\usepackage{array}
\usepackage{colortbl}
\usepackage{changebar}
\usepackage{amsmath}
\usepackage[T1]{fontenc}
%\usepackage[ansinew]{inputenc}
\usepackage{setspace}
\usepackage{multicol}
\usepackage{dcolumn}
\usepackage{color}
\usepackage{wasysym}
\usepackage{movie15}
\usepackage{MnSymbol}
\usepackage{marvosym}
\usepackage{pifont}
%\usepackage[table]{xcolor} 
%\beamerdefaultoverlayspecification{<+->}
\setbeamertemplate{footline}%{infolines theme}



\begin{document}

\lstset{
basicstyle=\ttfamily,
keywordstyle=\bfseries,
showstringspaces=false,
columns = fullflexible,
mathescape = false,
language=R
}


\title[Gallop Network Formation]
{Network Formation: Computation, Comparison, and Bargaining}
\author[M. Gallop]
{Max Gallop} 
\institute
{		Department of Political Science\\
		Duke University\\
		
		}

\date{Version 1.0}

\begin{frame}
\titlepage
\end{frame}


\begin{frame}{Finding Stable Networks}
\begin{itemize}
\item I now have Python code that allows, given a utility function and network with appropriate endowments, to find the Pairwise Nash Stable Networks!
\item The algorithm:
\begin{enumerate}
\item Iterate over all nodes, for each nodes empty link $ij$ add it if both i and j get more utility. Repeat until you get through all nodes without making changes.
\item Iterate over all nodes reducing each nodes link to the subset with the highest utility. Repeat until you iterate fully with no changes. If you made any changes go back to step 1, else you have a stable network.
\end{enumerate}
\end{itemize}
\end{frame}

\begin{frame}{A Toy Network}
To test the pairwise nash code I created a 8 ``state" toy network assigning each of them a capability and ideology score between 0 and 1.
\begin{table}
    \begin{tabular}{|l|l|l|}
        \hline
        ~       & Capability & Ideology \\ \hline
        US      & 1          & 1        \\ 
        UK      & .6         & 1        \\ 
        Germany & .8         & 1        \\ 
        France  & .4         & .9       \\ 
        Japan   & .8         & 1        \\ 
        China   & .8         & .2       \\ 
        Russia  & .5         & .2       \\ 
        Iran    & .1         & 0        \\
        \hline
    \end{tabular}
\end{table}
\begin{equation}
u_{i}(G) = \sqrt{\sum_{j \in G} (\text{Cap}_{j} * (1-|\text{Ideo}_{i} - \text{Ideo}_{j}) * \delta^{l(i,j)})} - d(i,G)c
\end{equation}
\end{frame}

\begin{frame}[shrink]{Example 1: Cost is 0}
\pause
\includegraphics{NoCost.png}
\end{frame}

\begin{frame}[shrink]{Example 1: Cost is 0.3, $\delta$ = .99}
\includegraphics{HighExternHighCost.png}
\end{frame}

\begin{frame}[shrink]{Example 1: Cost is 0.1, $\delta$ = .5}
\includegraphics{MidExternMidCost.png}
\end{frame}

\begin{frame}[shrink]{Example 1: Cost is 0.1, $\delta$ = .99}
\includegraphics{HighExternLowCost.png}
\end{frame}

\begin{frame}{Thoughts on Theoretical Mapping}
\begin{itemize}
\item Can formulate utility functions to correspond to different paradigms concerning cooperation:
\item For instance, the naive realist view that capability is all that matters, cost of cooperation is high, and positive externalities for cooperation are low.
\item Or a naive liberal view that ideology matters much more than capability, cost of cooperation is low, and large positive externalities.
\item Can formulate a number of these and have alternative predicted models of behavior. How do we compare them to reality?
\end{itemize}
\end{frame}

\begin{frame}{Thoughts on Empirical Mapping}
\begin{itemize}
\item Use data to create nodes and then find one more stable nash equilibrium given a certain utility function.
\item Compare to an observed network of cooperation using relevant graph statistics (from Hoff et al's work on ERGM models):
\begin{enumerate}
\item Geodesic Distance: The distribution of dyad shortest distance in the networks (ie what percent of dyads have shortest path 1 edge, 2,...)
\item Edgewise Shared Partner Distribtion: Number of links between dyads with 1 neighbor in common, 2 neighbors in common...
\item Degree distribution: Number of nodes with outdegree of exactly 0, 1...
\item Triad census distribution: Proportion of triads with 0, 1, 2,  or 3 edges.
\end{enumerate}
\item See whether any models of network formation outperform certain default models (a bernoulli random graph, an empty graph, a full graph) and if so which ones perform best.
\end{itemize}
\end{frame}

\begin{frame}{Cooperation is Nice, but Bargaining/Conflict is way more fun: Featuring a Block of Text Max Speeds Through}
\begin{itemize}
\item Treat bargaining/war as a direceted network formation game. Each state has a capability and an ideal policy.
\item If the network is empty each node gets utility based on the difference between a negotiated settlement at policy point $x$ and their preferred policy.
\item But any actor can ruin this by unilaterally making a link (I may want to limit who a state can fight to avoid weird actions where actors dragoon allies into an inevitable war).
\item If there are any links, there is a war between all actors with at least one link, and one of them wins with probability $\frac{c_{i}}{\sum_{j} c_{j}}$. States in the war network pay a cost, everyne gets utility based on the expected policy outcome.
\item If there exists a negotiated settlement where the empty network is an equilibrium, then we consider the bargaining network peaceful, else it is war prone.
\end{itemize}
\end{frame}
\end{document}
